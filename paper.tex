\documentclass[10pt,sigconf]{iabart}
%\documentclass[10pt,sigconf,letterpaper]{acmart}
%\usepackage[vskip=1em,font=itshape,leftmargin=2em,rightmargin=2em]{quoting}
\usepackage[skip=4pt plus1pt]{parskip}
%\copyrightyear{2024}
\acmYear{2024}
%\setcopyright{acmlicensed}\acmConference[IAB Network Management WS]{The IAB Next Era of Network Management Operations}{December 3, 2024}{}
%\acmBooktitle{The IAB next Era of Network Management Workshop}{December 3, 2024}{}

\begin{document}


\title{Evolving Challenges and Solutions in Network Management}
% Define short authors for page header
\shortauthors{Jiménez et al}

\author{Jaime Jiménez}
\email{jaime.jimenez@ericsson.com}
\orcid{0009-0008-2864-4269}
\affiliation{\small %
  \institution{Ericsson, ER}
  \city{Jorvas}
  \country{Finland}
}

\author{Scott Mansfield}
\email{scott.mansfield@ericsson.com}
\affiliation{\small %
  \institution{Ericsson, BNEW TS ST}
  \city{Jorvas}
  \country{Finland}
}

\author{Raquel Rodriguez A}
\email{raquel.a.rodriguez@ericsson.com}
\affiliation{\small %
  \institution{Ericsson, ETAC}
  \city{Jorvas}
  \country{Finland}
}

\author{Mikko Pesonen}
\email{mikko.pesonen@ericsson.com}
\affiliation{\small %
  \institution{Ericsson, ETAC}
  \city{Jorvas}
  \country{Finland}
}

\author{Vesa Torvinen}
\email{Vesa.Torvinen@ericsson.com}
\affiliation{\small %
  \institution{Ericsson, ETAC}
  \city{Jorvas}
  \country{Finland}
}

\author{Janne Karvonen}
\email{janne.karvonen@ericsson.com}
\affiliation{\small %
  \institution{Ericsson, ETAC}
  \city{Jorvas}
  \country{Finland}
}


\begin{abstract}

This paper presents an analysis of the evolving challenges and emerging solutions in network management. We explore scalability issues in network models and protocols, emphasizing the some limitations of current implementations the need for standardized, extendable information models. Telemetry complexities are discussed, focusing on data quality, diversity, and lineage, and the necessity for efficient data streaming and standardized schemas. Security challenges are addressed in the context of diverse protocols and the transition towards zero-trust architectures, highlighting the importance of unified security mechanisms and continuous updates. We also examine the future of network management, including the integration of generative AI and agentic architectures that adhere to autonomic networking principles, as well as the possibilities of new standard interfaces like CoRECONF. Our findings highlight the imperative for scalable, secure, and interoperable solutions that can adapt to the dynamic demands of modern telecommunications networks.

\end{abstract}

\keywords{Network Management, Scalability, Telemetry, Security, AI, Zero-Trust, Interoperability}

\maketitle

\section{Introduction} \label{introduction}

The IAB workshop on the Next Era of Network Management Operations (NEMOPS) serves as a platform for discussion between network operators, protocol experts and the general network management community. This workshop is expected to guide the Internet Engineering Task Force (IETF) standards process. The workshop's primary objectives are to assess past achievements and delineate future requirements for network management operations.

In this paper, we introduce a comprehensive analysis of the current challenges and emerging solutions in network management from the point of view of an SDN controller product. The subsequent sections delve into various aspects of network management and operations. The \textbf{Overall Architecture} section \ref{overview} provides a detailed overview of the standard components within a network management controller, the Ericsson Transport Automation Controller (ETAC). The \textbf{Scalability} section \ref{scalability} examines the challenges of scaling network models and protocols, highlighting the need for standardized models. The \textbf{Telemetry} section \ref{telemetry} discusses the complexities of data transmission. The \textbf{Security} section \ref{security} raises some security challenges and explains the shift towards zero-trust architectures. Finally, the \textbf{Network Management Evolution} section \ref{insights} explores potential future trends, including the role of generative AI and new standard interfaces.

\subsection{ETAC overal Architecture} \label{overview}

Ericsson Transport Automation Controller (ETAC) ~\cite{ericsson-etac} is a cloud-native Transport Automation and SDN Controller that leverages artificial intelligence and machine learning to deliver advanced analytics and automation functionalities across microwave, IP, and optical fronthaul networks.

\begin{figure}[h]
  \centering
  \includegraphics[width=0.5\textwidth]{figs/arch.pdf}
  \caption{General Architecture}
  \label{fig:overall_architecture}
\end{figure}

The \textbf{Ericsson Transport Automation Controller (ETAC)} \cite{ericsson-etac} adheres to the Open Transport SDN Reference Architecture as delineated by the TIP MUST project, aligning with the principles outlined in the Open Transport Architecture Whitepaper \cite{open_transport_architecture_whitepaper}. ETAC supports the roles of both the SDN Domain Controller and the SDN Hierarchical Controller, in accordance with the objectives of the Open Optical \& Packet Transport (OOPT) initiative \cite{oopt}. At high-level, it has two main Integration parts the Northound Integration part (NBI) and the Southbound Integration part (SBI) both of which use well-known standard interfaces and protocols like \textbf{NETCONF/YANG} \cite{RFC6241, rfc6020}, \textbf{RESTCONF} \cite{RFC8040}, \textbf{SNMP} \cite{RFC1157}, \textbf{SFTP} \cite{RFC4253} or \textbf{HTTP/S} \cite{RFC7230} towards APIs.

On the SBI it communicates with the managed nodes and it also is capable of translating between specific device information models and the harmonized, standards-based information model used in a network database. The Network Intelligence layer builds on top of this harmonized model, implementing the analytics, automation and SDN control application supported in the system. 

On the NBI it offers an exposure layer to users over a friendly GUI as well as to various OSS applications, providing access both to the network data and to the functionality supported in the system and enabling integration with other platforms. 

ETAC supports both SBI and NBI through implementations that accommodate legacy protocols and information models, whether defined by SOD or specific to vendors. It also provides real-time network observability, facilitating network analytics and closed-loop automation. It currently supports use cases that utilize its built-in real-time observability features to gain network insights through AI/ML and implement closed-loop automation in Transport Networks, all within a zero-trust framework.

\section{Scalability} \label{scalability}

% SCOTT please check this
Scalability in network models, protocols, devices, and systems is a complex issue. For example, optical equipment can have up to 10,000 interfaces, but current models like YANG often fall short because they use file databases without indexing. This highlights the need for standardized network-level models that can be easily mapped to devices, improving scalability across different network architectures. Additionally, integrating legacy systems is a significant challenge, as many network nodes still rely on SNMP, which calls for adaptable and scalable solutions. The difference between stateful and stateless connections adds another layer of complexity to scalability efforts.
 

\textbf{YANG Schema Mount}

Scalability in YANG presents significant challenges, as highlighted by Boyd \cite{boyd2023scalable}. The concern is about the viability of the current implementation of YANG Schema Mount \cite{RFC8528} and the need to support a mechanism that is better suited for partitioning the data in the hierarchical data tree.

% TODO: This needs to be rewritten more clearly
Even an enhanced solution for this issue will really only be solved by a better underlying modeling implementation including ACID/transactional, time-series, object-relational views with a query-language and the ability to add triggers or embedded code that becomes part of the solution. The complexity of YANG is intensifying, largely driven by its hierarchical architecture.

\textbf{Efficient Data Streaming for Analytics}

The efficient and scalable transfer of analytics data from its source to post-processing systems remains a significant challenge. Legacy network elements predominantly rely on periodic data harvesting, which imposes unnecessary load on the network elements and delays data accessibility. To address this, data sources should implement active streaming of data to post-processing systems immediately upon production. This approach ensures that post-processing systems and closed-loop automation have access to data in near real-time.


\section{Telemetry} \label{telemetry}


% TODO: BRIEF Intro
% The challenges of using UDP for telemetry.
% Need for fine-grained control over data transmission.
% Handling of mmulti-protocol data and reporting mechanisms. Traps, YANG push, webhooks
% YANG to CBOR dictionary for compression and the implications of frequent notifications.
% Signaling using lightweight protocols

% TODO: IMO the data part is to heavy with content specific for ETAC. We sould need some high-level general issues.
% SCOTT, please input here.

\textbf{Quality of data} 

The alignment of analytics data schemas and metadata is essential for efficient post-processing and analytics. Minor differences, such as variations in timestamp formats, can pose challenges. The adoption of standardized, extendable schemas and encodings for various analytics data types would facilitate streamlined data processing across different vendor systems.

Legacy network elements produce data that may not be fully compatible with modern IT-style post-processing analytics systems. To enhance compatibility, it is beneficial for this data to include embedded metadata providing details such as data originator.

Lastly, while the reliability of telemetry data may be less critical for visualization and manual inspections, it becomes essential when utilized for AI training and assisted network configuration.

\textbf{Diversity of data} 

Analytic data is often accessed through a variety of interfaces and standards. This variety presents challenges in streamlining data collection and ingestion processes. This is also true for including metrics which come in various forms like events, alarms, notifications, publications webhooks, logs, and other resource state data.

There is therefore need for an efficient, scalable, and universal transfer system to stream analytics data from its source to post-processing systems.

\textbf{Lineage of data}

It is essential to establish the lineage and integrity of any record used for purposes beyond visualization. Utilizing data with uncertain integrity can introduce new attack vectors, such as enabling adversaries to exploit closed-loop automation by manipulating the input data. 

Ensuring data integrity across multiple domains is crucial, especially when analytics data traverses cloud provider PaaS functions before reaching post-processing systems.

\textbf{State-Data Handling}

% TODO: This needs to be rewritten more clearly
In passive data sources where analytics data is just exposed (as opposite to streamed) the data has to be actively harvested by external entities, thus there must be APIs or interfaces for collection of such data. In systems where data is streamed, the data sources (often involved in critical functions such as traffic handling) do not need to expose interfaces for harvesting analytics data, which makes the attack surface smaller and system more secure in general.

Modern network elements have a lot of state-data which can not be efficiently leveraged through config model notifications. There are components and elements in the applications/data sources, which one does not to expose in a model, but still the analytics data about the state of such components should be available for post-processing systems. For example, one ideally would like to keep the model of cloud service realization agnostic, and use the same model for the cloud service, whether the application-instances of those services run in virtual machines or in container. Having to model compute resources in order to convey the state data would make it very tricky to separate realization and model from each other, and keep the model backwards compatible as the application realizations evolve and technologies change.

% \section{Query Range} \label{queryrange} 

% % Importance of protocol-agnostic network modeling, which allows for a broad query range.
% % Database storage and retrieval. Time-series storage. 
% % The meeting discussed the differences between UML-defined models and those used in YANG, focusing on how these models can operate in a protocol-neutral manner.

% TBD!


% TODO: Update Security by ETAC

\section{Security challenges} \label{security}

Security configuration in network management is complex due to the absence of a unified infrastructure. This complexity arises from the need to support multiple security protocols across diverse devices and vendors. Albeit rare, some challenges include:

\begin{itemize}
    \item \textbf{Diverse Protocols:} Network management applications must accommodate various security protocols, such as TLS/DTLS, SSH, and username/password mechanisms.
    \item \textbf{TLS/DTLS:} TLS/DTLS, particularly with client-server certificates, is a promising candidate for unified security. However, some vendors opt for alternative security techniques, such as VPNs.
    \item \textbf{SSH Limitations:} There is no security infrastructure available that would distribute and help verifying the SSH public keys and facilitate flexible re-keying.The security configuration of SSH itself remains to be manual. % verify this statement there are things like hashicorp https://www.vaultproject.io
    \item \textbf{Username/Password:} Commonly used for basic access and protocols like SNMPv3. Centralized authentication exists but cannot fully replace local authentication due to potential unavailability. Manual configuration often results in poor security practices, as key renewals and password updates are prone to errors and costly.

\end{itemize}
 
\textbf{Legacy security}

The transition from standards to widespread network deployment is often very slow, particularly when new standards are to replace existing components. Additionally, management applications often necessitate market adoption or commitment to the new mechanism before its implementation is considered worthwhile. Consequently, phasing out legacy security mechanisms can be challenging.
 
\textbf{Zero-Trust Architecture}

Ericsson develops products that are configured to use secure protocols and configurations by default. But we still see in legacy networks that sometimes the transport network is assumed to be secured, trusted, so that a lower degree of security is sometimes accepted. This may work in some networks but the trend is definitely towards zero-trust. 

Regulators and local legislation is also setting new requirements for critical infrastructure, such as telecom networks. The interest is not only on the security capabilities required from the network devices but also development processes, documentation, FOSS usage, security assurance and various aspects of the way how networks are deployed and managed.

One unfortunate reality is that some network devices in life networks are actually very old. If the network operates correctly, any upgrade or change to it may start looking like a risk. This may lead to neglection of software and security upgrades. This could work in the walled-garden paradigm but definitely not anymore in zero-trust. Continuous deployments and upgrades are the only path to truly secure networks. And all this would need to be automated in secure way, covering the software upgrades, re-configurations and testing.

 
\section{Network Management Evolution} \label{insights}

The evolution of network management could be influenced by advancements in AI and machine learning, as well as potential retrofitting of other protocols intended for constrained environments. These developments promise enhanced automation and scalability for telecommunications networks.

\textbf{AI Agents}

Zero-shot LLMs are are already used to make operator insights more accessible, to streamline incident management or to generate task-specific code from natural language queries. As the industry progresses in the recent GenAI path, agentic architectures will be integrated into network AI systems, adhering to the autonomic networking principles specified in RFC7575 \cite{RFC7575}.

\begin{quote}
"The fundamental goal is self-management, including self-configuration, self-optimization, self-healing, and self-protection." 
\end{quote}

Agentic systems are characterized by AI agents that are capable of decomposing user intents into executable individual steps and can interact autonomously with other systems through well-known interfaces (e.g., Network and Management APIs). Agents can further federate and specialize in multi-agent systems, which improve on challenges such as hallucinations, specialization, and scalability. The interest in multi-agent systems was reflected during the past IETF 121 side meeting, as detailed in the ai4network agenda \cite{ai4network-agenda}. 

\begin{figure}[h]
  \centering
  \includegraphics[width=0.5\textwidth]{figs/agent.pdf}
  \caption{Agent Architecture}
  \label{fig:agent_architecture}
\end{figure}

Agents are particularly useful in the telecommunications sector, where the complexity of specifications and codebases demands innovative solutions. Moreover, the current trend in 5G Networks is marked by a shift towards exposing network functionalities through APIs. This trend facilitates the integration of agentic systems that can dynamically interact with these APIs.

\textbf{CoRECONF}

CoRECONF, in particular,is utilized by constrained devices in Low-Power and Lossy Networks, which are typically composed of numerous embedded devices with limited power and memory. 

The the Constrained Application Protocol (CoAP) \cite{RFC7252}, a component of CoRECONF, is primarily utilized in IoT applications, it has also been specified for signaling DDoS-related telemetry \cite{RFC9132} and \cite{RFC9362}, as documented by the now-concluded DOTS Working Group. 

Within the network management community, the -CONF ecosystem is predominantly characterized by the widespread deployment of NETCONF and RESTCONF for network management tasks. In contrast, CoRECONF is not as widely know. The main differences being the application protocol and the serializations:

\begin{itemize}
  \item \textbf{NETCONF} \cite{RFC6241}: Serializing YANG over a stateful TCP connection.    
  \item \textbf{RESTCONF} \cite{RFC8040}: Serializing YANG over stateless HTTP.
  \item \textbf{CoRECONF} \cite{draft-ietf-core-comi}: Serializing YANG modules in a CBOR \cite{RFC9254} map over stateless CoAP.
\end{itemize}

This suggests potential applications for retrofitting CoAP on telemetry or management-type signaling within the network management domain. Specially in UDP-oriented evinronments as CoAP comes with an additional reliability mechanism and in environments where compression and smaller payloads are welcomed. 

\section{Conclusions} \label{conclusions}

% TODO: improve

This paper has explored the evolving landscape of network management, highlighting the challenges and solutions in scalability, telemetry, and security. The integration of advanced technologies such as AI and machine learning within platforms like the Ericsson Transport Automation Controller (ETAC) demonstrates the potential for enhanced analytics and automation in network operations. However, the complexity of legacy systems and the diversity of data sources present significant hurdles that require innovative approaches, such as standardized schemas and efficient data streaming mechanisms.

The transition towards zero-trust architectures and the adoption of generative AI indicate a shift towards more secure and autonomous network management systems. These advancements necessitate a collaborative effort between industry stakeholders and standardization bodies to ensure seamless integration and widespread adoption. As the network management domain continues to evolve, the focus must remain on developing scalable, secure, and interoperable solutions that can adapt to the dynamic demands of modern telecommunications networks.

\section{Acknowledgments}

% TODO: improve
We would like to thank Ericsson for their support of this work. Special thanks to ...


\bibliographystyle{ACM-Reference-Format}
\bibliography{paper}

\end{document}
\endinput
